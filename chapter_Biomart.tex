
%----------------------------------------------------------------------------------------
%	CHAP Biomart
%----------------------------------------------------------------------------------------

\chapterimage{blue-chapter-head_4-reduced.pdf} % Chapter heading image
\chapter{Biomart}\label{chap:Biomart}
%upload images for examples
%invert description and itemize statement

% introduction
\section{Overview}
The BioMart project develops software and data services that are made available to the international scientific community. Users of MetaR can access data Marts provided by BioMart, providing access to a wide range of research data . Similarly to EdgeR and Limma, BioMart support is provided in a MetaR language extension. This means that in order to access BioMart with MetaR, you first need to add the Biomart language to the model where you create the analysis. To do so, you can press~\keys{\ctrl+L} and add language \textit{org\allowbreak.campagnelab\allowbreak.metar\allowbreak.biomart}. After adding the biomart language to the model, you can create \texttt{query biomart} statements, described in the following sections. 

\section{The Biomart Statement}
The \texttt{query biomart} statement makes it possible to interactively specify which data should be retrieved from a mart (using attributes and filters). Data downloaded from Biomart will be stored in a new table. Figure~\ref{fig:NewBiomart} presents a newly created Biomart statement.
The \texttt{query biomart} statement has the following parameters.
\begin{itemize}
\item database
\item dataset
\item attributes
\item filters
\end{itemize} 

 \begin{figure}[h!tbp]
  \centering
  \includegraphics[width=\figWidthWide]{figures/NewBiomart.pdf}
\caption[New Query Biomart Statement.]{\textbf{New Query Biomart Statement.}}
\label{fig:NewBiomart}
\end{figure}


\paragraph{database}
The first time you will create the \texttt{query biomart} statement, you will automatically download the available Biomart databases. To choose one of them, go to the text \textit{select a database}. Then, press \keys{\ctrl+\space} to display all available databases. Select one of them by pressing \keys{\return}. 


 \paragraph{dataset} 
Once a database is selected, you need to choose a dataset. To choose one, press \keys{\ctrl+\space} to display all available datasets, on the text \textit{select a dataset}. Then, press \keys{\return}, to select your dataset. \newline
A dataset as two childrens:
\begin{itemize}
\item attributes, which will be the column of your result table.
\item filters, which allow you to restrict your result table for some criteria.
\end{itemize}
\begin{remark}
Sometimes, a dataset cannot be associated with attributes or filters. In this case, in the auto completion menu for both attributes and filters will display the message \textit{"No available filter or attributes in this dataset"}. The selected dataset is no more available in Biomart. You must change your dataset or database. 
\end{remark}

\paragraph{attributes}
Attributes are columns of your result table. Figure~\ref{fig:attributeBiomart} presents a new biomart attribute. An attribute has three parameters:
\begin{itemize}
\item attribute, a column you want to retrieve on your result table. Press \keys{\ctrl+\space} to display the available column.
\item type, the value type of the attribute. It exist three  value types
boolean, numeric and string (\textit{ by default}). To change the type, press \keys{\ctrl+\space}.
\item column group, an attribute can have a group such as "ID". The user can display the autocompletion menu by pressing \keys{\ctrl+\space}. Group must be defined in the Column Group Container to be added to the result table.
\end{itemize}

\begin{remark}
You must select at least one column to run your query biomart Statement.
\end{remark}

 \begin{figure}[h!tbp]
  \centering
  \includegraphics[width=\figWidthWide]{figures/BiomartAttribute.pdf}
\caption[Select an attribute in a biomart dataset.]{\textbf{Select an attribute in a biomart dataset.} An attribute contains any information you want to retrieve on your result table. It has a name, a type and a column group annotation. Attributes are related to a specific dataset.}
\label{fig:attributeBiomart}
\end{figure}
\paragraph{Filters}
Filters allow you to restrict your result for some criteria. this criteria are link directly to your dataset. Figure~\ref{fig:BiomartFilter} presents a new biomart filter. It exist four filter categories: 
\begin{itemize}
\item boolean, you want to retrieve or excluded only data which have an information. \textit{For example, if this gene has or has not a Mirbase ID.}
\item text, you have to enter a text in this area. the query will return only informations which match your text. \textit{For example  a specific GO term.}
\item list,  you need to choose a value from a define biomart list. \textit{For example a specific chromosome in a specie.}
\item id list, you can restrict your results from a set of ids. This ids can come from a set of ids define before the statement, or directly from an annotated table with an (ID group).  
\end{itemize}

 \begin{figure}[h!tbp]
  \centering
  \includegraphics[width=\figWidthWide]{figures/BiomartFilter.pdf}
\caption[New Biomart Filter]{\textbf{New Biomart Filter.} A filter allow you to apply constraints to you results. It exist four filters categories: boolean, text,list and id list. Filters are related to a specific dataset.}
\label{fig:BiomartFilter}
\end{figure}

\paragraph{table}
The future table where your result will be stored. Column annotation is display under the Inspector Tab (\inspectorTabIcon).
\section{Examples}
\subsection{Example 1}
Figure~\ref{fig:BiomartExample1} show how to obtain a table from the Ensembl database and Human dataset. The result table "resultFromBiomart" contains 2 columns, the Ensembl Gene and Exon ID, where the first column is annotated as a group "ID". This result are filters from a list of gene and which do not contains a Mirbase ID.
 \begin{figure}[h!tbp]
  \centering
  \includegraphics[width=\figWidthWide]{figures/BiomartExample1.pdf}
\caption[Biomart Example 1]{\textbf{ Biomart Example 1.}}
\label{fig:BiomartExample1}
\end{figure}

\subsection{Example 2}
Figure~\ref{fig:BiomartExample2} show how to obtain a table from the Paramecium bibliography. The result table "resultFromBiomart" contains 2 columns, the PubMed ID and the abstract, where the publication year is upper 2000.
 \begin{figure}[h!tbp]
  \centering
  \includegraphics[width=\figWidthWide]{figures/BiomartExample2.pdf}
\caption[Biomart Example 2]{\textbf{ Biomart Example 2.}}
\label{fig:BiomartExample2}
\end{figure}
