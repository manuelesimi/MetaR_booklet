
%----------------------------------------------------------------------------------------
%	The MPS Key Map
%----------------------------------------------------------------------------------------

\chapterimage{blue-chapter-head_4-reduced.pdf} % Chapter heading image

\chapter{Introduction}\label{chap:Introduction}
\section{Background}
The MetaR software \url{http://MetaR.campagnelab.org} is an example of a new kind of interactive tool for data analysis. It was developed by my laboratory using the Meta Programming System (MPS) (see \url{http://www.jetbrains.com/mps}~\cite{Dmitriev:2004,campagne2014mps}. MPS is a mature Language Workbench that makes it relatively easy to create new languages and tools to help users of these languages. 

\section{Intended audience}
This booklet is designed to teach how to use MetaR for data analysis. In the first chapters, I will assume that you have no prior scripting or programming experience, but will expect you to know how to use a computer.

Chapters~\ref{chap:Intentions}and~\ref{chap:ExtendingMetaR} will be useful for users who also have programming experience. These chapters explain how such users can extend MetaR with intentions or new language constructs.

\section{Key Concepts}
MetaR is designed to make it easier to conduct data analysis. To achieve this goal, I found it useful to define the following concepts:

\begin{enumerate}
	\item Table
	\item Column
	\item Column Group
	\item Column Group Usage
	\item Plot
	\item Model
\end{enumerate}
